\documentclass{beamer}
\usepackage{cancel}
\usepackage{amsthm}
\usepackage{amssymb}
\usepackage{amsmath}
\usepackage{mathtools}
\usepackage{bbold}
\usepackage{subfig}
\usepackage{tikz}
\usepackage{pgfplots}

\usetikzlibrary{decorations.pathmorphing}
\usetikzlibrary{arrows.meta}

\usepackage[ruled, linesnumbered]{algorithm2e}

\usepackage[sorting=none, url=false, isbn=false]{biblatex}

% Define the custom colors
%% Base color and direct complement
\definecolor{burntOrange}{HTML}{CC5500}
\definecolor{compl}{RGB}{51,170,255} % Complementary color to (204,85,0)
\definecolor{lightorange}{RGB}{255, 233, 217}

%% Earthy Complementary Palette
\definecolor{slateBlue}{HTML}{6699CC}
\definecolor{cream}{HTML}{FFF0D6}
\definecolor{charcoalBrown}{HTML}{331400}
\definecolor{warmGray}{HTML}{666666}

%% Warm Analogous Harmony
\definecolor{redOrange}{HTML}{CC3300}
\definecolor{goldenYellow}{HTML}{CC9900}
\definecolor{coffeeBrown}{HTML}{331A00}
\definecolor{antiqueWhite}{HTML}{F5E6CC}

%% Muted Earth Tones
\definecolor{oliveGreen}{HTML}{556B2F}
\definecolor{mustardYellow}{HTML}{CCAA00}
\definecolor{terracotta}{HTML}{994433}
\definecolor{linen}{HTML}{FFF8E7}

%% Monochromatic Depth
\definecolor{palePeach}{HTML}{FFD699}
\definecolor{spiceBrown}{HTML}{803300}
\definecolor{clay}{HTML}{B34700}

%% Triadic Boldness
\definecolor{teal}{HTML}{00CC99}
\definecolor{royalPurple}{HTML}{6600CC}
\definecolor{mintFrost}{HTML}{99E5CC}
\definecolor{lavenderMist}{HTML}{CCB3FF}

% Change the color theme
\setbeamercolor{normal text}{bg=antiqueWhite!25} % Background
\setbeamercolor{structure}{fg=white, bg=oliveGreen}
\setbeamercolor{title}{fg=white, bg=burntOrange}
\setbeamercolor{block body}{bg=lightorange}
\setbeamercolor{frametitle}{fg=white, bg=burntOrange}
\setbeamercolor{section in toc}{fg=burntOrange}
\setbeamercolor{itemize item}{fg=burntOrange}
\setbeamercolor{enumerate item}{fg=burntOrange}
\setbeamercolor{bibliography item}{fg=burntOrange}
\setbeamercolor{bibliography entry author}{fg=burntOrange}
\setbeamercolor{bibliography entry note}{fg=burntOrange}

\setbeamercolor{palette quaternary}{fg=black, bg=palePeach}

\AtBeginSection[]{
\begin{frame}
\vfill \centering
\begin{beamercolorbox}[sep=8pt, center,rounded=true]{palette quaternary}%
\usebeamerfont{title}\insertsectionhead\par%
\end{beamercolorbox}
\vfill
\end{frame}
}


\newcommand\defeq{\coloneqq}

% ------
%\newtheorem{theorem}{Theorem}
%\newtheorem{corollary}{Corollary}[theorem]
%\newtheorem{lemma}[theorem]{Lemma}
\newtheorem{proposition}[theorem]{Proposition}
\newtheorem*{lemma*}{Lemma}
\newtheorem*{proposition*}{Proposition}
% ------


\newcommand{\dial}[1]{\tikz{\draw[thick] (0,0) circle (1em);\draw[thick,->] (0,0) -- ++(90-#1:0.8em);}}


\title{Unstructured Adiabatic Quantum Optimization: Optimality with Limitations}
\author{Arthur Braida \and Shantanav Chakraborty \and Alapan Chaudhuri \and \textbf{Joseph
Cunningham} \and Rutvij Menavlikar \and Leonardo Novo \and Jérémie Roland}
\date{}
\setbeamertemplate{footline}[frame number]


\DeclarePairedDelimiter{\norm}{\lVert}{\rVert}
% Differential (upface d)
\DeclareMathOperator{\diff}{d \!}
% Derivative (upface D)
\DeclareMathOperator{\Diff}{D \!}
% ordinary derivative - analogous to the partial derivative command
\providecommand{\od}[3][]{\ensuremath{
\ifinner
\tfrac{\diff{^{#1}}#2}{\diff{{#3}^{#1}}}
\else
\dfrac{\diff{^{#1}}#2}{\diff{{#3}^{#1}}}
\fi
}}

\providecommand{\tod}[3][]{\ensuremath{\mathinner{
\tfrac{\diff{^{#1}}#2}{\diff{{#3}^{#1}}}
}}}
\providecommand{\dod}[3][]{\ensuremath{\mathinner{
\dfrac{\diff{^{#1}}#2}{\diff{{#3}^{#1}}}
}}}
\DeclareMathOperator{\Tr}{Tr}

\newcommand{\ket}[1]{\left| #1 \right\rangle}
\newcommand{\bra}[1]{\left\langle #1 \right|}
\let\abraket\braket
\newcommand{\braket}[3][\null]{%    %TOreDO!
  \ifx#1\null
       \langle#2|#3\rangle%
    \else%
       \langle#2|#1|#3\rangle%
    \fi}
\newcommand{\ketbra}[2]{|#1\rangle\langle#2|}

\begin{document}

\frame{\titlepage \vspace{-3em} \begin{figure}
\begin{minipage}[t]{.3\textwidth}
  \centering
  \includegraphics[height=4.5em]{logoULBsmall}
\end{minipage}%
\begin{minipage}[t]{.3\textwidth}
  \centering
  \includegraphics[height=4.5em]{logoIIIT}
\end{minipage}
\begin{minipage}[t]{.3\textwidth}
  \centering
  \includegraphics[height=4.5em]{logoINL}
\end{minipage}
\end{figure}\url{arXiv:2411.05736}}
\begin{frame}
\frametitle{Table of contents}
\tableofcontents
\end{frame}

\section{The Ising model}
\begin{frame}
\frametitle{The Ising model}

\begin{tikzpicture}[scale=2]
    % Define grid size
    \def \rows {5}
    \def \cols {6}
    \def \dx {0.8} % Horizontal step
    \def \dy {0.5} % Vertical step
    
    % Loop through the grid with perspective
    \foreach \x in {1,...,\cols} {
        \foreach \y in {1,...,\rows} {
            % Compute coordinates for perspective
            \pgfmathsetmacro\xa{\x*\dx + \y*\dx*0.5}
            \pgfmathsetmacro\ya{\y*\dy}
            
            
            % Draw the bonds
            \ifnum \x<\cols
                \pgfmathsetmacro\xb{(\x+1)*\dx + \y*\dx*0.5}
                \pgfmathsetmacro\yb{\y*\dy}
                \draw[dashed] (\xa,\ya) -- (\xb,\yb);
                
                \onslide<2->{\pgfmathsetmacro\colorPercent{random(0,1)*100}
                \pgfmathsetmacro\rand{random(0,100)}
                \pgfmathsetmacro\opacity{ifthenelse(\rand<=70,1,0)}
                \draw[thick,decorate,decoration={coil,aspect=1.8,segment length=2mm,amplitude=1mm},color=green!\colorPercent!red, opacity=\opacity] 
                    (\xa,\ya) -- (\xb,\yb);}
                
            \fi
            \ifnum \y<\rows
                \pgfmathsetmacro\xu{\x*\dx + (\y+1)*\dx*0.5}
                \pgfmathsetmacro\yu{(\y+1)*\dy}
                \draw[dashed] (\xa,\ya) -- (\xu,\yu);
                
                \onslide<2->{\pgfmathsetmacro\colorPercent{random(0,1)*100}
                \pgfmathsetmacro\rand{random(0,100)}
                \pgfmathsetmacro\opacity{ifthenelse(\rand<=70,1,0)}
                \draw[thick,decorate,decoration={coil,aspect=1.8,segment length=2mm,amplitude=1mm},color=green!\colorPercent!red, opacity=\opacity] 
                    (\xa,\ya) -- (\xu,\yu);}
            \fi
            
            
            % Assign spins (alternating up/down pattern)
            \pgfmathtruncatemacro\randSpin{random(0,1)}
            \ifnum\randSpin=0
                \draw[->,thick] (\xa,\ya) --++ (0,0.3);
            \else
                \draw[->,thick] (\xa,\ya) --++ (0,-0.3);
            \fi
        }
    }
    
    \foreach \x in {1,...,\cols} {
        \foreach \y in {1,...,\rows} {
            \pgfmathsetmacro\xa{\x*\dx + \y*\dx*0.5}
            \pgfmathsetmacro\ya{\y*\dy}
            \fill (\xa,\ya) circle(0.05);
        }
    }

\end{tikzpicture}
\end{frame}

\begin{frame}
\frametitle{The Ising model: 2-colouring a graph}

\begin{tikzpicture}[scale=2]
    % Define grid size
    \def \rows {5}
    \def \cols {6}
    \def \dx {0.8} % Horizontal step
    \def \dy {0.5} % Vertical step
    
    % Loop through the grid with perspective
    \foreach \x in {1,...,\cols} {
        \foreach \y in {1,...,\rows} {
            % Compute coordinates for perspective
            \pgfmathsetmacro\xa{\x*\dx + \y*\dx*0.5}
            \pgfmathsetmacro\ya{\y*\dy}
            
            
            % Draw the bonds
            \ifnum \x<\cols
                \pgfmathsetmacro\xb{(\x+1)*\dx + \y*\dx*0.5}
                \pgfmathsetmacro\yb{\y*\dy}
                \draw[dashed] (\xa,\ya) -- (\xb,\yb);
            \fi
            \ifnum \y<\rows
                \pgfmathsetmacro\xu{\x*\dx + (\y+1)*\dx*0.5}
                \pgfmathsetmacro\yu{(\y+1)*\dy}
                \draw[dashed] (\xa,\ya) -- (\xu,\yu);
            \fi
            
        }
    }
    
    \pgfmathsetmacro\x{2}
    \pgfmathsetmacro\y{2}
    \draw[thick] (\x*\dx + \y*\dx*0.5, \y*\dy) --++ (\dx, 0);
    \only<3>{\draw[thick,decorate,decoration={coil,aspect=1.8,segment length=2mm,amplitude=1mm},color=red] 
                    (\x*\dx + \y*\dx*0.5, \y*\dy) --++ (\dx, 0);}
    \pgfmathsetmacro\x{3}
    \pgfmathsetmacro\y{2}
    \draw[thick] (\x*\dx + \y*\dx*0.5, \y*\dy) --++ (\dx, 0);
    \only<3>{\draw[thick,decorate,decoration={coil,aspect=1.8,segment length=2mm,amplitude=1mm},color=red] 
                    (\x*\dx + \y*\dx*0.5, \y*\dy) --++ (\dx, 0);}
    \pgfmathsetmacro\x{3}
    \pgfmathsetmacro\y{3}
    \draw[thick] (\x*\dx + \y*\dx*0.5, \y*\dy) --++ (\dx, 0);
    \only<3>{\draw[thick,decorate,decoration={coil,aspect=1.8,segment length=2mm,amplitude=1mm},color=red] 
                    (\x*\dx + \y*\dx*0.5, \y*\dy) --++ (\dx, 0);}
    \pgfmathsetmacro\x{3}
    \pgfmathsetmacro\y{4}
    \draw[thick] (\x*\dx + \y*\dx*0.5, \y*\dy) --++ (\dx, 0);
    \only<3>{\draw[thick,decorate,decoration={coil,aspect=1.8,segment length=2mm,amplitude=1mm},color=red] 
                    (\x*\dx + \y*\dx*0.5, \y*\dy) --++ (\dx, 0);}
    
    \pgfmathsetmacro\x{3}
    \pgfmathsetmacro\y{2}
    \draw[thick] (\x*\dx + \y*\dx*0.5, \y*\dy) --++ (0.5*\dx, \dy);
    \only<3>{\draw[thick,decorate,decoration={coil,aspect=1.8,segment length=2mm,amplitude=1mm},color=red] 
                    (\x*\dx + \y*\dx*0.5, \y*\dy) --++ (0.5*\dx, \dy);}
    \pgfmathsetmacro\x{4}
    \pgfmathsetmacro\y{2}
    \draw[thick] (\x*\dx + \y*\dx*0.5, \y*\dy) --++ (0.5*\dx, \dy);
    \only<3>{\draw[thick,decorate,decoration={coil,aspect=1.8,segment length=2mm,amplitude=1mm},color=red] 
                    (\x*\dx + \y*\dx*0.5, \y*\dy) --++ (0.5*\dx, \dy);}
    \pgfmathsetmacro\x{3}
    \pgfmathsetmacro\y{3}
    \draw[thick] (\x*\dx + \y*\dx*0.5, \y*\dy) --++ (0.5*\dx, \dy);
    \only<3>{\draw[thick,decorate,decoration={coil,aspect=1.8,segment length=2mm,amplitude=1mm},color=red] 
                    (\x*\dx + \y*\dx*0.5, \y*\dy) --++ (0.5*\dx, \dy);}
    \pgfmathsetmacro\x{4}
    \pgfmathsetmacro\y{3}
    \draw[thick] (\x*\dx + \y*\dx*0.5, \y*\dy) --++ (0.5*\dx, \dy);
    \only<3>{\draw[thick,decorate,decoration={coil,aspect=1.8,segment length=2mm,amplitude=1mm},color=red] 
                    (\x*\dx + \y*\dx*0.5, \y*\dy) --++ (0.5*\dx, \dy);}
    
    \foreach \x in {1,...,\cols} {
        \foreach \y in {1,...,\rows} {
            \pgfmathsetmacro\xa{\x*\dx + \y*\dx*0.5}
            \pgfmathsetmacro\ya{\y*\dy}
            \fill (\xa,\ya) circle(0.05);
        }
    }
    
    \pgfmathsetmacro\x{2}
    \pgfmathsetmacro\y{2}
    \only<2>{\fill[yellow] (\x*\dx + \y*\dx*0.5, \y*\dy) circle(0.05);}
    \pgfmathsetmacro\x{3}
    \pgfmathsetmacro\y{2}
    \only<2>{\fill[blue] (\x*\dx + \y*\dx*0.5, \y*\dy) circle(0.05);}
    \pgfmathsetmacro\x{3}
    \pgfmathsetmacro\y{3}
    \only<2>{\fill[yellow] (\x*\dx + \y*\dx*0.5, \y*\dy) circle(0.05);}
    \pgfmathsetmacro\x{3}
    \pgfmathsetmacro\y{4}
    \only<2>{\fill[blue] (\x*\dx + \y*\dx*0.5, \y*\dy) circle(0.05);}
    \pgfmathsetmacro\x{4}
    \pgfmathsetmacro\y{2}
    \only<2>{\fill[yellow] (\x*\dx + \y*\dx*0.5, \y*\dy) circle(0.05);}
    \pgfmathsetmacro\x{4}
    \pgfmathsetmacro\y{3}
    \only<2>{\fill[blue] (\x*\dx + \y*\dx*0.5, \y*\dy) circle(0.05);}
    \pgfmathsetmacro\x{4}
    \pgfmathsetmacro\y{4}
    \only<2>{\fill[yellow] (\x*\dx + \y*\dx*0.5, \y*\dy) circle(0.05);}
    
\end{tikzpicture}
\end{frame}

\begin{frame}
\frametitle{The Ising model: the Hamiltonian}
\begin{tikzpicture}[scale=2]
    % Define grid size
    \def \rows {5}
    \def \cols {6}
    \def \dx {0.8} % Horizontal step
    \def \dy {0.5} % Vertical step
    
    % Loop through the grid with perspective
    \foreach \x in {1,...,\cols} {
        \foreach \y in {1,...,\rows} {
            % Compute coordinates for perspective
            \pgfmathsetmacro\xa{\x*\dx + \y*\dx*0.5}
            \pgfmathsetmacro\ya{\y*\dy}
            
            
            % Draw the bonds
            \ifnum \x<\cols
                \pgfmathsetmacro\xb{(\x+1)*\dx + \y*\dx*0.5}
                \pgfmathsetmacro\yb{\y*\dy}
                \draw[dashed] (\xa,\ya) -- (\xb,\yb);
                
                \pgfmathsetmacro\colorPercent{random(0,1)*100}
                \pgfmathsetmacro\rand{random(0,100)}
                \pgfmathsetmacro\opacity{ifthenelse(\rand<=70,1,0)}
                \draw[thick,decorate,decoration={coil,aspect=1.8,segment length=2mm,amplitude=1mm},color=green!\colorPercent!red, opacity=\opacity] 
                    (\xa,\ya) -- (\xb,\yb);
                
            \fi
            \ifnum \y<\rows
                \pgfmathsetmacro\xu{\x*\dx + (\y+1)*\dx*0.5}
                \pgfmathsetmacro\yu{(\y+1)*\dy}
                \draw[dashed] (\xa,\ya) -- (\xu,\yu);
                
                \pgfmathsetmacro\colorPercent{random(0,1)*100}
                \pgfmathsetmacro\rand{random(0,100)}
                \pgfmathsetmacro\opacity{ifthenelse(\rand<=70,1,0)}
                \draw[thick,decorate,decoration={coil,aspect=1.8,segment length=2mm,amplitude=1mm},color=green!\colorPercent!red, opacity=\opacity] 
                    (\xa,\ya) -- (\xu,\yu);
            \fi
        }
    }
    
    \foreach \x in {1,...,\cols} {
        \foreach \y in {1,...,\rows} {
            \pgfmathsetmacro\xa{\x*\dx + \y*\dx*0.5}
            \pgfmathsetmacro\ya{\y*\dy}
            \fill (\xa,\ya) circle(0.05);
        }
    }

\end{tikzpicture}

\begin{equation*}
H = \sum_{\langle i,j \rangle} J_{ij} \sigma_z^{i} \sigma_z^{j} + \sum_{j=1}^{n}h_j\sigma_z^{j}
\end{equation*}
where $J_{ij},~h_j \in \{-1,0,1\}$.
\end{frame}

\begin{frame}
\frametitle{Preparing the ground state of an Ising Hamiltonian}
\begin{itemize}[<+->]
\item Computationally interesting (NP-complete)
\item Straightforward on a circuit-based quantum computer
\item A physics problem (adiabatic methods seem natural)
\end{itemize}
\end{frame}

\begin{frame}
\frametitle{The Ising model: adiabatic quantum computation}

\begin{figure}
\begin{minipage}[t]{.5\textwidth}
  \centering
  \begin{tikzpicture}[scale=1]
    % Define grid size
    \def \rows {5}
    \def \cols {6}
    \def \dx {0.8} % Horizontal step
    \def \dy {0.5} % Vertical step
    
    % Loop through the grid with perspective
    \foreach \x in {1,...,\cols} {
        \foreach \y in {1,...,\rows} {
            % Compute coordinates for perspective
            \pgfmathsetmacro\xa{\x*\dx + \y*\dx*0.5}
            \pgfmathsetmacro\ya{\y*\dy}
            
            
            % Draw the bonds
            \ifnum \x<\cols
                \pgfmathsetmacro\xb{(\x+1)*\dx + \y*\dx*0.5}
                \pgfmathsetmacro\yb{\y*\dy}
                \draw[dashed] (\xa,\ya) -- (\xb,\yb);
            \fi
            \ifnum \y<\rows
                \pgfmathsetmacro\xu{\x*\dx + (\y+1)*\dx*0.5}
                \pgfmathsetmacro\yu{(\y+1)*\dy}
                \draw[dashed] (\xa,\ya) -- (\xu,\yu);
            \fi
            
            \draw[->,thick, green] (\xa,\ya) --++ (0.4,0);
        }
    }
    \foreach \x in {1,...,\cols} {
        \foreach \y in {1,...,\rows} {
            \pgfmathsetmacro\xa{\x*\dx + \y*\dx*0.5}
            \pgfmathsetmacro\ya{\y*\dy}
            \fill (\xa,\ya) circle(0.05);
        }
    }
\end{tikzpicture}
\[ H_0 = \sum_{j=1}^n\sigma_x^j \]
\end{minipage}%
\begin{minipage}[t]{.5\textwidth}
  \centering
\begin{tikzpicture}[scale=1]
    % Define grid size
    \def \rows {5}
    \def \cols {6}
    \def \dx {0.8} % Horizontal step
    \def \dy {0.5} % Vertical step
    
    % Loop through the grid with perspective
    \foreach \x in {1,...,\cols} {
        \foreach \y in {1,...,\rows} {
            % Compute coordinates for perspective
            \pgfmathsetmacro\xa{\x*\dx + \y*\dx*0.5}
            \pgfmathsetmacro\ya{\y*\dy}
            
            
            % Draw the bonds
            \ifnum \x<\cols
                \pgfmathsetmacro\xb{(\x+1)*\dx + \y*\dx*0.5}
                \pgfmathsetmacro\yb{\y*\dy}
                \draw[dashed] (\xa,\ya) -- (\xb,\yb);
                
                \pgfmathsetmacro\colorPercent{random(0,1)*100}
                \pgfmathsetmacro\rand{random(0,100)}
                \pgfmathsetmacro\opacity{ifthenelse(\rand<=70,1,0)}
                \draw[thick,decorate,decoration={coil,aspect=1.5,segment length=1.2mm,amplitude=0.6mm},color=green!\colorPercent!red, opacity=\opacity] 
                    (\xa,\ya) -- (\xb,\yb);
                
            \fi
            \ifnum \y<\rows
                \pgfmathsetmacro\xu{\x*\dx + (\y+1)*\dx*0.5}
                \pgfmathsetmacro\yu{(\y+1)*\dy}
                \draw[dashed] (\xa,\ya) -- (\xu,\yu);
                
                \pgfmathsetmacro\colorPercent{random(0,1)*100}
                \pgfmathsetmacro\rand{random(0,100)}
                \pgfmathsetmacro\opacity{ifthenelse(\rand<=70,1,0)}
                \draw[thick,decorate,decoration={coil,aspect=1.5,segment length=1.2mm,amplitude=0.6mm},color=green!\colorPercent!red, opacity=\opacity] 
                    (\xa,\ya) -- (\xu,\yu);
            \fi
        }
    }
    \foreach \x in {1,...,\cols} {
        \foreach \y in {1,...,\rows} {
            \pgfmathsetmacro\xa{\x*\dx + \y*\dx*0.5}
            \pgfmathsetmacro\ya{\y*\dy}
            \fill (\xa,\ya) circle(0.05);
        }
    }
\end{tikzpicture}
\[ H_1 = \sum_{\langle i,j \rangle} J_{ij} \sigma_z^{i} \sigma_z^{j} + \sum_{j=1}^{n}h_j\sigma_z^{j} \]
\end{minipage}
\end{figure}

\onslide<2->{\[ H(s) = (1-s)H_0 + sH_1 \]}
\onslide<3->{In this work:}
\begin{itemize}
\item<4-> $H_0 = - \ketbra{u}{u}$ ($\ket{u}$ is uniform superposition)
\item<5-> Generalisation of $H_1$
\end{itemize}

\end{frame}

\section{Adiabatic quantum computing}
\begin{frame}
\frametitle{Adiabatic quantum computing}

\begin{algorithm}[H]
Prepare system in the ground state of $H(0)$\;
Apply $H(0)$\;
Slowly change $s$ from $0$ to $1$ in the applied Hamiltonian $H(s)$\;
\caption{Adiabatic quantum computing}
\end{algorithm}

\vspace{2em}

We call the function $s(t)$ the \textit{schedule}.
\end{frame}

\begin{frame}
\frametitle{The spectrum and the gap}
Suppose $H(s)$ is a Hamiltonian for $s\in [0,1]$
\[\begin{tikzpicture}[scale=6, every node/.style={font=\small}]

  % Parameters for the avoided crossing (bottom two curves)
  \def\Eavg{0.4}      % average level (center of the avoided crossing)
  \def\Vval{0.04}     % coupling strength; minimal gap = 2*\Vval = 0.08
  \def\kval{0.6}      % controls the "linear" splitting away from the crossing
  %\def\Delta{0.08}    % minimal gap (at x = 0.5), equal to 2*\Vval
  
  % Parameter for the wavy upper lines
  \def\A{0.03}       % amplitude for the sine waves

  % Draw coordinate axes
  \draw[->] (-0.05,0) -- (1.05,0) node[right] {$s$};
  \draw[->] (0,-0.05) -- (0,1) node[above] {$E$};

  % Label x-axis endpoints: "0" (moved slightly left) and tick/label at "1"
  \node[below] at (-0.03,0) {0};
  \draw (1, -0.01) -- (1, 0.01);
  \node[below] at (1,0) {1};

  % --------------------------
  % Bottom two curves: avoided crossing
  % The eigenvalue functions for a 2x2 system with off-diagonal coupling V are
  %   E_\pm(x) = E_{\text{avg}} \pm \sqrt{(k(x-0.5))^2 + V^2}.
  
  % Lower branch
  \draw[domain=0:1, smooth, variable=\x, compl, thick]
    plot ({\x}, { \Eavg - sqrt((\kval*(\x-0.5))^2 + (\Vval)^2) });
    
  % Upper branch
  \draw[domain=0:1, smooth, variable=\x, compl, thick]
    plot ({\x}, { \Eavg + sqrt((\kval*(\x-0.5))^2 + (\Vval)^2) });
    
  % Draw a dashed double arrow at x=0.5 showing the minimal gap, labeled Δ.
  % At x=0.5, (x-0.5)=0 so the values are E_avg ± V.
  \draw[<->, >=stealth, dotted] (1, { \Eavg - sqrt((\kval*(1-0.5))^2 + (\Vval)^2) + 0.01 }) -- (1, { \Eavg + sqrt((\kval*(1-0.5))^2 + (\Vval)^2) - 0.01 }) node[midway, right] {$g$};

  % --------------------------
  % Upper three curves: random wavy lines (they do not intersect)
  % All lines use the same complementary color.
  
  \draw[domain=0:1, smooth, variable=\x, compl, thick] 
    plot ({\x}, {0.75 + \A*sin(360*\x + 150)});
    
  \draw[domain=0:1, smooth, variable=\x, compl, thick] 
    plot ({\x}, {0.80 + \A*sin(360*\x + 220)});
    
  \draw[domain=0:1, smooth, variable=\x, compl, thick] 
    plot ({\x}, {0.85 + \A*sin(360*\x + 290)});

\end{tikzpicture}\]
\end{frame}


\begin{frame}
\frametitle{A trade-off between speed and accuracy}
For fixed precision $\epsilon$, the complexity scales as
\begin{itemize}
\item<1-> $O\big(\frac{\norm{H'}}{g_m^2}\big)$ (naive bound)
\only<1>{\item[] \phantom{$O\big(\frac{\norm{H'}}{g_m^2} + \frac{\norm{H'}^2}{g_m^3} + \frac{\norm{H''}}{g_m^2}\big)$}}
\only<2>{\item $O\big(\frac{\norm{H'}}{g_m^2} + \frac{\norm{H'}^2}{g_m^3} + \frac{\norm{H''}}{g_m^2}\big)$ (general bound)}
\only<3->{\item $O\big(\frac{\norm{H'}}{g_m^2} + \int_0^1\frac{\norm{H'}^2}{g^3} + \frac{\norm{H''}}{g^2}\diff{s}\big)$ (general bound)}
\item<4-> $O\big(\frac{1}{g_m}\big)$ (with adaptive scheduling)
\end{itemize}
\vspace{1em}
$g_m$ is the minimal gap.
\end{frame}

\subsection{Grover}
\begin{frame}
\frametitle{Some examples}
\begin{itemize}
\item<1-> \textbf{Grover search} Let $P$ be a projector. Prepare a state in the subspace associated to $P$. \onslide<2->{Let $\ket{u}$ be the uniform superposition. Then
\[ H(s) = (1-s)\big(\mathbb{1} - \ketbra{u}{u}\big) + s \big(\mathbb{1} - P\big) \]}
\item<3-> \textbf{Quantum Linear Systems Problem (QLSP)} Let $A\in \mathbb{C}^{N\times N}$ be an invertible matrix and $b\in \mathbb{C}^N$ a vector. Prepare the quantum state $\frac{A^{-1}\ket{b}}{\norm{A^{-1}\ket{b}}}$.
\end{itemize}
\end{frame}

\begin{frame}
\frametitle{The gap}

\begin{figure}
\begin{minipage}[t]{.5\textwidth}
  \centering
  \includegraphics[width=\textwidth]{GroverGap}
  Grover
  
  \onslide<2->{$\min_s g(s) = \Omega(\sqrt{1/N})$}
\end{minipage}%
\begin{minipage}[t]{.5\textwidth}
  \centering
  \includegraphics[width=\textwidth]{QLSPGap}
  QLSP
  
  \onslide<2->{$\min_s g(s) = \Omega(\kappa)$ where $\kappa = \norm{A}\norm{A^{-1}}$}
\end{minipage}
\end{figure}

\onslide<3->{In both cases
\[ \int_0^1 \frac{1}{g^p} \diff{s} = O\Big(\frac{1}{\min_s g^{p-1}}\Big) \]
for all $p>1$.
}
\end{frame}

\section{Analysis of the spectrum}

\begin{frame}
\frametitle{The spectrum}
\[ H(s) = -(1-s)\ketbra{u}{u} + s H_1 \]
\centering
 \includegraphics[width=\textwidth]{plot_ACinAQC002}
\end{frame}

\begin{frame}
\frametitle{The spectrum}
\centering
 \includegraphics[width=\textwidth]{plot_ACinAQC004}
\end{frame}

\begin{frame}
\frametitle{Some spectral quantities}
Given a spectral decomposition
\[ H_1 = \sum_{k} E_k P_k \]
with $0\leq E_0 < E_1 < E_2 < \ldots \leq 1 $ and $\operatorname{dim}(P_k) = d_k$, \pause we define
\begin{itemize}
\item $\Delta \coloneqq E_1 - E_0$
\item $A_1 \coloneqq \frac{1}{2^n}\sum_{k\neq 0} \frac{d_k}{E_k - E_0}$
\item $A_2 \coloneqq \frac{1}{2^n}\sum_{k\neq 0} \frac{d_k}{(E_k - E_0)^2}$
\end{itemize}
\pause

We assume
\[ \Delta \geq 100 \sqrt{\frac{d_0}{2^n A_2}} \]
\end{frame}

\begin{frame}
\frametitle{The gap}
The characteristic equation for the eigenvalues of
\[ H(s) = -(1-s)\ketbra{u}{u} + s H_1 \]
is
\[ 0 = -1 + \frac{1-s}{2^n}\sum_k \frac{d_k}{sE_k - \lambda(s)}. \]
\pause

\begin{tikzpicture}[scale=0.8]
    \begin{axis}[
        width=14cm, height=7cm,
        axis x line=middle,
        axis y line=middle,
        xmin=-1, xmax=10,
        ymin=-10, ymax=10,
        xtick={0},
        ytick={0},
        samples=100
    ]
        % Function plot: -1 + 1/(2-x) + 1/(3-x)
        \addplot[compl, domain=-1:10, samples=200, thick] {-1 + 0.5/(1-x) + 0.5/(3-x) + 0.5/(6-x) + 0.5/(7-x)};
        
        \node[anchor=south west] at (axis cs:1,0) {$sE_0$};
        \node[anchor=south west] at (axis cs:3,0) {$sE_1$};
        \node[anchor=south west] at (axis cs:6,0) {$sE_2$};
        \node[anchor=south west] at (axis cs:7,0) {$sE_3$};
    \end{axis}
\end{tikzpicture}
\end{frame}

\begin{frame}
\frametitle{The spectrum}
\[ H(s) = -(1-s)\ketbra{u}{u} + s H_1 \]
\centering
 \includegraphics[width=\textwidth]{plot_ACinAQC002}
\end{frame}


\begin{frame}
\frametitle{The gap}
\begin{figure}
\centering
 \includegraphics[width=0.6\textwidth]{plot_ACinAQC005}
 \end{figure}
\onslide<2->{$\min_s g = \frac{2A_1}{A_1+1}\sqrt{\frac{d_0}{2^nA_2}}$}
\onslide<3->{Again
\[ \int_0^1 \frac{1}{g^p} \diff{s} = O\Big(\frac{1}{\min_s g^{p-1}}\Big) \]
for all $p>1$}
\end{frame}

\subsection{A quadratic speedup}
\begin{frame}
\frametitle{A speedup}
We have a complexity $O(1/g_m^2)$, can we get $O(1/g_m)$?
\pause
\begin{theorem}
There exists a piecewise linear schedule $s(t)$, defined using $A_1,A_2$ and $\Delta$, such that the adiabatic procedure prepares an equal superposition of the ground states of $H_1$, with fidelity $1-\epsilon$ in time
\[ T = O\Big(\frac{1}{\epsilon}\cdot \frac{\sqrt{A_2}}{A_1^2\Delta^2}\cdot \sqrt{\frac{2^n}{d_0}}\Big). \]
\end{theorem}
\end{frame}

\begin{frame}
\frametitle{A sketch of the proof}
\begin{figure}
\centering
 \includegraphics[width=\textwidth]{plot2b}
 \end{figure}
\end{frame}

\begin{frame}
\frametitle{A sketch of the proof}
\begin{itemize}[<+->]
\item \textbf{Left of the avoided crossing}: variational principle with ansatz
\[ \ket{\phi} = \frac{1}{\sqrt{A_2 2^n}}\sum_k \frac{\sqrt{d_k}}{E_k - E_0}\ket{k} \]
\item \textbf{At the avoided crossing}: set $\lambda = sE_0 + \delta$ in the eigenvalue equation to obtain
\[ 0 = -1 + \frac{1-s}{2^n}\sum_k \frac{d_k}{s(E_k - E_0) - \delta} \]
and analyse for small $\delta$
\item \textbf{Right of the avoided crossing}: use the norm of the resolvent $g(s) \geq 2 \norm{(\gamma(s) - H(s))^{-1}}^{-1}$ with the Sherman-Morrison formula
\[ \big(A + \ketbra{u}{v}\big)^{-1} = A^{-1} - \frac{A^{-1}\ketbra{u}{v}A^{-1}}{1+\braket[A]{u}{v}} \]
\end{itemize}
\end{frame}

\section{Finding the location of the minimal gap is hard}
\begin{frame}
\frametitle{Locating the minimal gap}
The minimal gap is located at $s^* = \frac{A_1}{1+A_1}$
\pause

\begin{theorem}
Calculating $A_1$
\begin{itemize}
\item up to an error $O(1/n)$ is NP-hard.
\item up to an error $O(2^{-\operatorname{poly}(n)})$ is \#P-hard.
\end{itemize}
\end{theorem}
\end{frame}

\subsection{Hardness for non-local Hamiltonians}
\begin{frame}
\frametitle{Calculating $A_1$ for non-local Hamiltonians is hard}
\begin{itemize}[<+->]
\item Consider the problem of determining whether $E_0 = 0$ or $E_0 \neq 0$ for some Hamiltonian $H$
\item Apply the function
\[\begin{tikzpicture}
  \begin{axis}[
      axis lines = middle,
      xlabel = $x$,
      ylabel = {$\theta(x)$},
      xtick={-3,-2,-1,0,1,2,3},
      ytick={0,1},
      ymin=-0.2, ymax=1.8,
      xmin=-1.1, xmax=1.1,
      samples=2,
      domain=-1.1:1.1,
      width=10cm,
      height=3cm,
      clip=false
    ]
    % Plot for x < 0 (Heaviside value = 0)
    \addplot[compl,domain=-1.1:0,thick] {0};
    
    % Plot for x > 0 (Heaviside value = 1)
    \addplot[compl,domain=0:1.1,thick] {1};
    
    % Mark the discontinuity at x=0
    % Open circle at (0,0) to indicate exclusion
    \addplot[compl,only marks,mark=*,mark size=2pt] coordinates {(0,0)};
    % Filled circle at (0,1) to indicate inclusion
    \addplot[compl,only marks,mark=o,mark size=2pt] coordinates {(0,1)};
  \end{axis}
\end{tikzpicture}\]
i.e.\ do spectral flattening
\item (This can be approximated by polynomials)
\item If $E_0 = 0$, then $A_1 = 1 - \frac{d_0}{2^n}$
\item If $E_0 \neq 0$, then $A_1 = 0$
\end{itemize}

\onslide<6->{This is highly non-local!}
\end{frame}

\subsection{Approximating is still hard}
\begin{frame}
\frametitle{Calculating $A_1$ for local Hamiltonians is still hard}
\begin{proposition}
Let $\epsilon,~\mu_1, \mu_2\in (0,1)$. Suppose we have a procedure that outputs $\tilde{A}_1(H)$ such that
$$
\left|\tilde{A}_1(H)-A_1(H)\right|\leq \varepsilon.
$$
Suppose either (i)~$E_0=0$ or (ii)~$0\leq \mu_1\leq E_0\leq 1-\mu_2 \leq 1$. Then, it is possible to decide whether (i) or (ii) holds if
$$
\varepsilon < \dfrac{\mu_1}{6(1-\mu_1)}-\dfrac{d_0}{6~2^n}\cdot\dfrac{1}{\mu_1\mu_2}
$$
\end{proposition}
\pause
\textit{Proof.}
We have
\[ A_1(H)-2A_1\bigg(H\otimes \Big(\frac{1+\sigma_z}{2}\Big)\bigg) \quad \begin{cases}= 0 \\ \geq \dfrac{\mu_1}{1-\mu_1}-\dfrac{d_0}{2^n}\cdot\dfrac{1}{\mu_1\mu_2} \end{cases} \]
\end{frame}

\subsection{Finding it exactly is very hard}
\begin{frame}
\frametitle{Calculating $A_1$ almost exactly is very hard}
\begin{proposition}
Suppose $H$ be a $n$-qubit Ising Hamiltonian. Then it is possible to estimate the degeneracy $d_k$ of the energy eigenvalue $E_k$ by making $O(\mathrm{poly}(n))$ calls to $A_1$. 
\end{proposition}
\pause
\textit{Proof.} Define
\begin{align*}
f(x) &= A_1(H)-2A_1\bigg(H\otimes \mathbb{1} - x\mathbb{1}\otimes \Big(\frac{1+\sigma_z}{4}\Big)\bigg) \\
&= \frac{1}{2^n} \sum_{k = 0}\frac{d_k}{E_k - E_0 + x/2}
\end{align*}
and use Lagrange interpolation.
\end{frame}

\section{Conclusion}
\begin{frame}
\frametitle{Conclusion}
\begin{itemize}
\item<1-> We have shown a generic quadratic speedup over brute force search
\item<2-> Limited applicability if $s^*$ is not known
\end{itemize}
\onslide<3->{Open questions}
\begin{itemize}
\item<4-> Precise hardness of estimating $s^*$ to desired accuracy?
\item<5-> Hardness of estimating $s^*$ for $2$-local Hamiltonians?
\item<6-> What about $H_0 = \sum_{j}\sigma_x^j$?
\end{itemize}
\vspace{.5em}

\centering
\includegraphics[width=3cm]{qrcode}

\url{arXiv:2411.05736}

\end{frame}

\begin{frame}[noframenumbering, allowframebreaks]
\printbibliography
\end{frame}

\begin{frame}[noframenumbering]
\frametitle{Rabi oscillation}
\begin{itemize}
\item Two-level system with qubit frequency $\omega_0$ due to Zeeman splitting
\item High frequency ($\omega$), but low amplitude ($\omega_1$) transverse field
\end{itemize}

\[\begin{tikzpicture}
  % Energy levels
  \draw[thick] (0,0) -- (2,0) node[right] {$|0\rangle$};
  \draw[thick] (0,2) -- (2,2) node[right] {$|1\rangle$};
  
  % Transition frequency (ω₀)
  \draw[{<[scale=1.3]}-{>[scale=1.3]}, dashed] (1,0.05) -- (1,1.95) node[midway, right] {$\quad\omega_0$};
  
  % Driving field (wavy arrow with amplitude annotation)
  \draw[->, decorate, decoration={snake, amplitude=2mm, segment length=2.5mm}] (-1.5,1) -- (0.5,1)
    node[midway, above, yshift=4pt] {$\omega$};  % Raised frequency label
  \draw[<->, gray] (-1.7,0.78) -- (-1.7,1.22)    % Amplitude indicator
    node[midway, left] {$\omega_1$};           % Amplitude label
\end{tikzpicture}\]
\end{frame}

\begin{frame}[noframenumbering]
\frametitle{Rabi oscillation}
\[ H(s) = \begin{pmatrix}\omega_0 & \omega_1e^{-i\omega s} \\ \omega_1e^{i\omega s} & -\omega_0 \end{pmatrix} \qquad g = 2\sqrt{\omega_0^2 + \omega_1^2} \qquad \begin{aligned}\norm{H'} &= \omega_1\omega \\ \norm{H''} &= \omega_1\omega^2\end{aligned} \]

\[ \begin{tikzpicture}[xscale=5, yscale=1, every node/.style={font=\small}]

  % Draw coordinate axes
  \draw[->] (-0.05,0) -- (1.05,0) node[right] {$s$};
  \draw[->] (0,-0.05) -- (0,1.05) node[above] {$P(0\to 1)$};

  % Label x-axis endpoints: "0" (shifted left) and tick/label at "1"
  \node[below] at (-0.03,0) {0};
  \draw (1, -0.01) -- (1, 0.01);
  \node[below] at (1,0) {1};

  % Plot y = sin^2(6πx) in the complementary color.
  % Note: sin(1080*x) is used because 6π rad = 1080°.
  \draw[domain=0:1, smooth, variable=\x, compl, thick]
    plot ({\x}, {sin(500*\x)^2});

\end{tikzpicture} \qquad \raisebox{2em}{$\Omega = \sqrt{(\omega_0 - \omega)^2 + \omega_1^2}$} \]

\begin{align*}
\epsilon &= O\Big(\frac{\norm{H'}}{g_m^2} + \frac{\norm{H'}^2}{g_m^3} + \frac{\norm{H''}}{g_m^2}\Big) \\
&= O\Big(\frac{\omega_0\omega_1}{\omega_0^2 + \omega_1^2} + \frac{\omega_0^2\omega_1^2}{(\omega_0^2 + \omega_1^2)^{3/2}} + \frac{\omega_0^2\omega_1}{\omega_0^2 + \omega_1^2}\Big) \\
&\approx O\Big(\frac{1}{\omega_0} + \frac{1}{\omega_0} + \omega_1\Big) \\
\end{align*}
\end{frame}

\end{document}

