% Chapter 3: Quantum Computation
%
% OVERVIEW: Establishes quantum mechanics and computation background.
% Introduces Hilbert spaces, Hamiltonians, and the quantum speedup paradigm.
%
% NEEDS: Chapter 2 (Computation basics, complexity classes)
%
% FORWARD: Chapter 4 builds AQC on this foundation
% FORWARD: Chapter 5 uses Hamiltonians, eigenvalues, spectral decomposition
% FORWARD: Chapter 6 uses variational principle
%
% DEFINITIONS INTRODUCED: Hilbert space, qubit, Hamiltonian, eigenvalue,
% eigenvector, spectral gap, unitary, measurement, BQP
%
% NOTE: Avoid re-defining terms in later chapters

% =============================================================================
% Section: Quantum Mechanics Essentials
% =============================================================================
% SOURCE: Standard QM texts (Sakurai, Nielsen-Chuang)
%
% - States: vectors in Hilbert space, superposition
% - Measurement: Born rule, collapse, probabilistic outcomes
% - Unitary evolution: Schrodinger equation, reversibility
%
% Key concepts for thesis:
% - States are vectors |psi> in complex Hilbert space H
% - Probabilities: |<phi|psi>|^2 (Born rule)
% - Evolution: |psi(t)> = U(t)|psi(0)> where U is unitary
%
% TODO: Decide level of detail (physics vs CS audience)
% TODO: Emphasize concepts used later, not full QM course
%
% REVIEWER: presentation_quality - Accessible to CS reader

% =============================================================================
% Section: Hilbert Spaces and the Computational Basis
% =============================================================================
% SOURCE: Nielsen-Chuang Chapter 2
%
% Definitions:
% - Qubit: two-dimensional Hilbert space C^2, basis {|0>, |1|}
% - n-qubit system: tensor product (C^2)^{tensor n} = C^{2^n}
% - Computational basis: |x> for x in {0,1}^n, dimension N = 2^n
%
% Key properties:
% - Inner product: <phi|psi> (complex number)
% - Norm: ||psi||^2 = <psi|psi> = 1 for normalized states
% - Superposition: |psi> = sum_x alpha_x |x>, sum|alpha_x|^2 = 1
%
% For thesis:
% - N = 2^n is exponential in n (resource counting)
% - Computational basis is where H_z is diagonal (Chapter 5)
%
% TODO: Define notation clearly (bra-ket)
% TODO: Tensor product construction
%
% FORWARD: Chapter 5 uses N = 2^n, computational basis states

% =============================================================================
% Section: Hamiltonians as Energy
% =============================================================================
% SOURCE: Standard QM, Nielsen-Chuang
%
% Definitions:
% - Hermitian operator: H = H^dagger (self-adjoint)
% - Eigenvalues: real numbers E_k (energy levels)
% - Eigenvectors: states |k> with H|k> = E_k|k>
%
% Spectral decomposition:
% H = sum_k E_k |k><k| (for non-degenerate case)
% H = sum_k E_k P_k where P_k projects onto E_k eigenspace (degenerate)
%
% Ground state:
% - Lowest eigenvalue E_0, corresponding eigenstates
% - For optimization: ground state encodes solution
%
% Spectral gap:
% - Gap between ground and first excited: g = E_1 - E_0
% - Central to adiabatic computation (Chapter 4)
%
% TODO: Handle degeneracy properly (d_k-fold degeneracy)
% TODO: Emphasize spectral gap as key quantity
%
% NOTE: "Spectral gap" defined here; Chapter 4 discusses computational role
%
% FORWARD: Chapter 5 uses spectral decomposition of H_z
% FORWARD: Chapters 5-6 analyze spectral gap of H(s)
%
% REVIEWER: technical_soundness - Definitions must be precise

% =============================================================================
% Section: Time Evolution
% =============================================================================
% SOURCE: Nielsen-Chuang Chapter 2
%
% Schrodinger equation (time-independent H):
% i * d|psi>/dt = H|psi>
%
% Solution:
% |psi(t)> = e^{-iHt}|psi(0)> = U(t)|psi(0)>
%
% where U(t) = e^{-iHt} is unitary (preserves norm).
%
% Time-dependent H(t):
% - More complex: time-ordered exponential
% - Adiabatic case: H(s) with s = s(t)
% - Relevant for AQC (Chapter 4)
%
% TODO: Clarify units (hbar = 1 convention)
% TODO: Time-dependent case for AQC
%
% FORWARD: Chapter 4 uses time-dependent H(s)
%
% REVIEWER: technical_soundness - Units and conventions

% =============================================================================
% Section: Why Quantum Helps
% =============================================================================
%
% Quantum resources for computation:
%
% Superposition:
% - Process many inputs "simultaneously"
% - N = 2^n amplitudes in n qubits
% - NOT parallel computation (amplitudes interfere)
%
% Interference:
% - Amplitudes add (can cancel or reinforce)
% - Algorithm design: amplify correct answers, cancel wrong
% - Key to quantum speedups
%
% Tunneling (relevant for AQO):
% - Quantum systems can traverse classically forbidden regions
% - Potential escape from local minima
% - Whether this helps for optimization is debated
%
% Entanglement:
% - Correlations with no classical analog
% - Resource for quantum protocols
% - Less central to AQO (diagonal H_z has product ground states)
%
% TODO: Be honest about what is proven vs hoped
% TODO: Tunneling claims are controversial for optimization
%
% REVIEWER: significance - Motivate quantum computation

% =============================================================================
% Section: Circuits and Query Complexity
% =============================================================================
% SOURCE: Nielsen-Chuang, Arora-Barak
%
% Gate model:
% - Universal gate sets (H, T, CNOT)
% - Circuit depth and width
% - Polynomial circuits define BQP
%
% BQP (Bounded-error Quantum Polynomial time):
% - Problems solvable by polynomial-time quantum circuits
% - With probability >= 2/3
% - Believed: P subset BQP subset PSPACE
%
% Oracle model:
% - Black-box access to function f: {0,1}^n -> {0,1}
% - Quantum oracle: |x,y> -> |x, y XOR f(x)>
% - Query complexity: count oracle calls
%
% For thesis:
% - Grover uses O(sqrt(N)) queries (optimal)
% - AQO doesn't use explicit oracles, but H_z encodes f
%
% TODO: Define BQP precisely
% TODO: Oracle vs non-oracle models
%
% FORWARD: Chapter 7/8 compare AQO to circuit model
%
% REVIEWER: literature - Standard definitions

% =============================================================================
% Section: Grover as Geometry
% =============================================================================
% SOURCE: Grover (1996), Nielsen-Chuang
%
% The search problem:
% - Find marked item w among N items
% - Oracle: f(x) = 1 if x = w, else 0
%
% Geometric picture:
% - Two-dimensional subspace: span{|w>, |s>}
%   where |s> = (1/sqrt(N)) sum_x |x> (uniform superposition)
% - Grover iteration rotates in this plane
% - Angle per iteration: theta = arcsin(1/sqrt(N)) ~ 1/sqrt(N)
%
% Runtime:
% - Need pi/(4*theta) ~ sqrt(N) iterations to reach |w>
% - O(sqrt(N)) queries to oracle
%
% Why geometric view matters:
% - Makes optimality transparent
% - Connects to AQO: same two-level structure (Chapter 5)
%
% TODO: Draw the rotation picture
% TODO: Connect to Roland-Cerf (Chapter 4)
%
% FORWARD: Chapter 5: AQO has similar two-level structure near s*
%
% REVIEWER: presentation_quality - Geometry before algebra

% =============================================================================
% Section: Why Grover is Optimal
% =============================================================================
% SOURCE: Bennett-Bernstein-Brassard-Vazirani (1997)
%
% BBBV lower bound:
% Any quantum algorithm needs Omega(sqrt(N)) queries to find marked item
% (with constant probability).
%
% Proof idea:
% - Query complexity limited by how fast amplitude can move to marked state
% - Polynomial method: amplitude polynomial in query count
% - Degree sqrt(N) needed to distinguish marked from unmarked
%
% Implication:
% - Quadratic speedup is fundamental, not algorithmic accident
% - Cannot do better than sqrt(N) for unstructured search
% - AQO matching this is optimal (Chapter 7)
%
% TODO: Sketch proof or give intuition
% TODO: Connect to runtime lower bounds for AQO
%
% FORWARD: Chapter 7: AQO achieves O~(sqrt(N/d_0)), matching lower bound
%
% REVIEWER: technical_soundness - Lower bound is important

% =============================================================================
% Section: The Decoherence Challenge
% =============================================================================
%
% Practical issues:
% - Maintaining quantum states is hard
% - Environmental interaction causes decoherence
% - Superposition degrades to classical mixture
%
% Error correction:
% - Encode logical qubits in many physical qubits
% - Detect and correct errors
% - Threshold theorem: fault-tolerant quantum computation possible
%
% Why adiabatic methods might help:
% - Energy gap provides natural protection
% - Ground state is stable (cannot decay further)
% - But: must maintain gap throughout evolution
% - NOT a free lunch: gap can still be small
%
% For thesis:
% - We assume ideal evolution (no noise)
% - Noise analysis is beyond scope
% - Focus is on computational complexity, not physics
%
% TODO: Be clear about assumptions
% TODO: Note that decoherence is separate from computational hardness
%
% OPTION: Could expand or minimize depending on thesis scope
%
% REVIEWER: presentation_quality - Set expectations clearly
