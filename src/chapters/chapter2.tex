% Chapter 2: Physics and Computation
%
% OVERVIEW: First-principles foundation connecting physics to computation.
% Builds intuition before formalism. Sets stage for quantum computation.
%
% FORWARD: Chapter 3 (Quantum computation specifics)
% FORWARD: Chapter 4 (AQC as physics-computation bridge)
% FORWARD: Chapter 8 (NP, #P complexity classes for hardness)
%
% DEFINITIONS INTRODUCED: P, NP, NP-complete, NP-hard, #P, reduction
%
% NOTE: This chapter is conceptual; formal definitions in later chapters

% =============================================================================
% Section: Reality
% =============================================================================
%
% Setting the stage philosophically (brief):
%
% - Ontology: what exists (states, transformations, observables)
% - Epistemology: what we can know and predict
% - Mathematics as language: not reality, but model of reality
%
% For thesis:
% - Physics provides substrate for computation
% - Computation provides tools to analyze physics
% - The interaction is the subject
%
% TODO: Keep brief - this is not a philosophy thesis
% TODO: Set tone for mathematical rigor without losing intuition
%
% OPTION: Could be very brief or omitted
%
% REVIEWER: presentation_quality - Don't overdo philosophy

% =============================================================================
% Section: Physics
% =============================================================================
%
% Three pillars (brief overview):
%
% Classical mechanics:
% - Hamiltonian formulation: H(q,p) determines evolution
% - Phase space: state is (position, momentum)
% - Determinism: state at t=0 determines all future
%
% Statistical mechanics:
% - Entropy: measure of disorder/uncertainty
% - Equilibrium: systems tend toward low energy
% - Connection to optimization: finding ground state
%
% Quantum mechanics (preview - full treatment in Chapter 3):
% - Hilbert space: states are vectors
% - Unitarity: evolution preserves probability
% - Measurement: probabilistic collapse
%
% For thesis:
% - Hamiltonian formulation connects to quantum mechanics
% - Ground state as optimization target
% - Quantum mechanics enables new computational approaches
%
% TODO: Balance breadth vs depth
% TODO: Foreshadow quantum advantage
%
% FORWARD: Chapter 3 develops quantum mechanics fully

% =============================================================================
% Section: Computation
% =============================================================================
% SOURCE: Arora-Barak, Sipser
%
% What is computation?
% - Turing machines: formal model of computation
% - Church-Turing thesis: all "reasonable" models equivalent
% - Universality: single machine can simulate any other
%
% Computability:
% - Decidable vs undecidable problems
% - Halting problem: no algorithm can decide if programs halt
% - Limits of computation
%
% Complexity (key definitions for thesis):
%
% P: Problems solvable in polynomial time (efficient)
% - Time O(n^k) for some constant k
% - Considered "tractable"
%
% NP: Problems with efficiently verifiable solutions
% - Given solution, can check in polynomial time
% - Finding solution may be hard
%
% NP-complete: Hardest problems in NP
% - All NP problems reduce to them
% - If any NP-complete in P, then P = NP
% - Examples: SAT, 3-SAT, MAX-CUT
%
% NP-hard: At least as hard as NP-complete
% - May not be in NP (decision vs optimization)
%
% TODO: Define reduction precisely
% TODO: SAT and 3-SAT as canonical examples (used in Chapter 8)
%
% FORWARD: Chapter 8 uses NP-hardness for A_1 estimation
%
% REVIEWER: technical_soundness - Standard definitions

% =============================================================================
% Section: The #P Complexity Class
% =============================================================================
% SOURCE: paper/v3-quantum.tex lines 208-209, Valiant (1979)
%
% #P: Counting problems
% - Instead of "does solution exist?" (NP), ask "how many?"
% - Example: #SAT counts satisfying assignments
%
% #P-complete:
% - Hardest counting problems
% - At least as hard as any NP problem
% - #SAT is #P-complete
%
% Relation to thesis (Chapter 8):
% - Extracting degeneracy d_0 = counting ground states
% - For SAT Hamiltonian: d_0 = number of satisfying assignments
% - Computing A_1 exactly is #P-hard
%
% TODO: Define formally
% TODO: Explain why counting is harder than decision
%
% FORWARD: Chapter 8 uses #P-hardness for exact A_1
%
% REVIEWER: technical_soundness - Definition needed for hardness

% =============================================================================
% Section: Linearity and Its Limits
% =============================================================================
%
% Why linearity is powerful:
% - Superposition principle: solutions combine
% - Linear systems tractable: eigenvalue analysis, closed forms
% - Perturbation theory: small changes give small effects
%
% Quantum mechanics is exactly linear:
% - Schrodinger equation is linear in |psi>
% - Superposition of solutions is solution
% - Why quantum computing works differently from classical
%
% Nonlinearity and hardness:
% - Classical chaos: small changes amplify
% - Phase transitions: qualitative changes at critical points
% - Computational hardness often tied to nonlinear dynamics
%
% For thesis:
% - H(s) is linear in |psi>, but ground state changes nonlinearly in s
% - Avoided crossing is nonlinear phenomenon
% - Gap behavior determines computational complexity
%
% TODO: Connect linearity of QM to computational structure
% TODO: Preview gap as source of nonlinearity
%
% FORWARD: Chapter 5-6: gap analysis shows nonlinear behavior

% =============================================================================
% Section: Adiabaticity
% =============================================================================
% SOURCE: Physics textbooks, Griffiths QM
%
% Thermodynamic adiabatic:
% - Slow, reversible process
% - No heat exchange with environment
% - System stays in equilibrium
%
% Quantum adiabatic:
% - Slow evolution of Hamiltonian H(t)
% - System stays in instantaneous ground state
% - Key condition: "slow" means d/dt << gap^2
%
% The energy gap as key parameter:
% - Large gap: can evolve faster
% - Small gap: must evolve slower
% - Gap determines required time
%
% Landau-Zener transitions (preview):
% - What happens when you go too fast
% - Probability of jumping to excited state
% - Exponential suppression requires slowness ~ 1/gap^2
%
% TODO: Distinguish thermodynamic vs quantum adiabatic
% TODO: Give intuition for gap-slowness relationship
%
% FORWARD: Chapter 4 formalizes adiabatic theorem
%
% REVIEWER: presentation_quality - Physical intuition first

% =============================================================================
% Section: Energy Landscapes and Optimization
% =============================================================================
%
% Ground states as optimization solutions:
% - Minimize H means minimize energy
% - Ground state = optimal solution
% - Natural for optimization problems
%
% Complex landscapes:
% - Spin glasses: many local minima, frustration
% - NP-hard problems have rugged landscapes
% - Classical methods get stuck
%
% Classical optimization methods:
% - Simulated annealing: random thermal fluctuations
% - Gradient descent: follow local slope
% - Limitations: local minima, exponential time for hard instances
%
% Quantum approach (preview):
% - Quantum tunneling through barriers
% - Adiabatic evolution tracks global minimum
% - But: gap determines speed, small gap = slow
%
% For thesis:
% - AQO finds ground state of diagonal H_z
% - Question: how fast compared to classical?
% - Answer: Grover speedup, but with caveats (Chapters 7-8)
%
% TODO: Connect to specific problems (SAT, MAX-CUT)
% TODO: Be honest about what quantum helps with
%
% FORWARD: Chapter 5 formalizes AQO setup

% =============================================================================
% Section: Bridge to Quantum Computation
% =============================================================================
%
% Quantum speedup sources:
% - Superposition: many states at once
% - Interference: amplify correct, cancel wrong
% - Entanglement: non-classical correlations
%
% BQP and quantum complexity (preview):
% - BQP: problems solvable efficiently on quantum computers
% - Relationship to classical classes: P, NP, PSPACE
% - Quantum advantage for some problems
%
% Adiabatic quantum computation (preview):
% - Encode problem in Hamiltonian
% - Evolve slowly from easy to hard
% - Gap determines runtime
%
% Central question of thesis:
% - Can AQO achieve provable speedup for unstructured search?
% - Answer: Yes, but with information requirements
%
% TODO: Foreshadow the main results
% TODO: Set expectations for Chapters 4-9
%
% FORWARD: Chapters 3-4 develop quantum computation and AQC
%
% REVIEWER: significance - Connect to thesis contribution
