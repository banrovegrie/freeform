% Chapter 4: Adiabatic Quantum Computation
%
% OVERVIEW: This chapter introduces AQC and sets up the framework for AQO.
% Bridges from general quantum computation to the specific AQO setting.
%
% NEEDS: Chapter 3 (Hamiltonians, eigenvalues, time evolution, Grover)
% NEEDS: Chapter 2 (Complexity classes, optimization)
%
% FORWARD: Chapter 5 specializes to AQO with diagonal H_z
% FORWARD: Chapter 7 uses Roland-Cerf construction for schedule
%
% DEFINITIONS INTRODUCED: adiabatic theorem, spectral gap (as computational resource),
% avoided crossing, local schedule
%
% NOTE: "spectral gap" as energy difference is in Chapter 3;
% here we emphasize it as computational bottleneck

% =============================================================================
% Section: Physical Motivation
% =============================================================================
% SOURCE: Standard quantum mechanics texts (Sakurai, Griffiths)
%
% - Adiabatic theorem: slow evolution stays in ground state
% - The gap determines how slow is slow enough
% - Physical intuition: system tracks changing potential
% - Analogy: pendulum length changed slowly vs quickly
%
% TODO: Give concrete physical example (magnetic field rotation)
% TODO: Distinguish "adiabatic" in quantum vs thermodynamic sense
%
% REVIEWER: presentation_quality - Physical intuition before formalism

% =============================================================================
% Section: The Computational Idea
% =============================================================================
% SOURCE: Farhi et al. (2001) original AQC proposal
%
% - Encode problem in Hamiltonian H_z: ground state is solution
% - Start with easy Hamiltonian H_0: known ground state
% - Interpolate: H(s) = (1-s)H_0 + sH_z for s in [0,1]
% - Read answer: measure at s=1
%
% This is general AQC setup. Chapter 5 specializes:
% - H_0 = -|psi_0><psi_0| (projector)
% - H_z diagonal in computational basis
%
% TODO: Explain encoding of optimization problems (MAX-SAT, etc.)
% TODO: Why this is natural for optimization
%
% FORWARD: Chapter 5 uses H(s) = -(1-s)|psi_0><psi_0| + sH_z

% =============================================================================
% Section: The Adiabatic Theorem
% =============================================================================
% SOURCE: Born-Fock (1928), Kato (1950), Jansen-Ruskai-Seiler (2007)
%
% Statement: If evolution is slow enough, stay in instantaneous ground state.
%
% Quantitative form (Jansen-Ruskai-Seiler):
% Error <= C * max_s |dH/ds| / min_s g(s)^2
%
% where g(s) = lambda_1(s) - lambda_0(s) is the spectral gap.
%
% Implications:
% - Runtime scales as inverse gap squared for worst case
% - Gap minimum dominates: T ~ 1/g_min^2
%
% Note: Chapter 7 uses refined version with local schedule
%
% TODO: State theorem precisely with all conditions
% TODO: Distinguish from original adiabatic theorem (qualitative)
% TODO: Reference Cunningham et al. for phase randomization extension
%
% FORWARD: Chapter 7 uses this for runtime derivation
%
% REVIEWER: technical_soundness - Must state conditions carefully

% =============================================================================
% Section: The Spectral Gap as Central Bridge
% =============================================================================
%
% The gap connects physics to complexity:
%
% Physics perspective:
% - Large gap: robust ground state, fast evolution
% - Small gap: fragile ground state, slow evolution required
%
% Computational perspective:
% - Large gap: easy computation
% - Small gap: hard computation (or at least slow)
%
% The gap as bottleneck:
% - Runtime ~ 1/g_min^2 (from adiabatic theorem)
% - Small gap => exponential time
% - QMA-hardness comes from estimating gap (Kitaev's local Hamiltonian)
%
% TODO: Connect to QMA and local Hamiltonian problem
% TODO: Gap as "complexity witness"
%
% FORWARD: Chapters 5-6 analyze gap structure for AQO
%
% REVIEWER: significance - Gap is the central object of the thesis

% =============================================================================
% Section: Avoided Crossings
% =============================================================================
% SOURCE: Landau (1932), Zener (1932)
%
% Definition: Point where energy levels approach but don't cross
% (eigenvalues would cross for some parameter but repel due to coupling)
%
% Landau-Zener formula:
% Transition probability P ~ exp(-pi * g_min^2 / (2 * v))
% where v = d(E_1 - E_0)/dt at crossing
%
% Implications:
% - Must slow down near crossings
% - Gap minimum typically occurs at avoided crossing
% - Single avoided crossing simplifies analysis (AQO case)
%
% TODO: Derive Landau-Zener or give intuition
% TODO: Explain why eigenvalues repel (non-crossing theorem)
%
% FORWARD: Chapter 5: AQO has single avoided crossing at s*
% FORWARD: Chapter 6: Gap bounds in three regions around crossing
%
% REVIEWER: presentation_quality - Intuition before formula

% =============================================================================
% Section: Schedules
% =============================================================================
%
% Linear schedule: s(t) = t/T
% - Simple but not optimal
% - Spends equal time everywhere
% - Wastes time where gap is large
%
% Adaptive schedule: ds/dt proportional to g(s)^2
% - Slow where gap is small
% - Fast where gap is large
% - Concentrates time at bottleneck
%
% Gap-aware vs gap-unaware:
% - Gap-aware: knows g(s), can adapt
% - Gap-unaware: must work for all instances
%
% TODO: Compare runtimes for linear vs adaptive
% TODO: Set up the gap-uninformed problem (Chapter 9)
%
% FORWARD: Chapter 7 constructs optimal schedule
% FORWARD: Chapter 9 analyzes gap-uninformed case

% =============================================================================
% Section: Roland-Cerf Construction
% =============================================================================
% SOURCE: Roland-Cerf (2002)
%
% The first AQC algorithm achieving Grover speedup:
%
% Setup:
% - H_0 = -|s><s| where |s> = uniform superposition
% - H_z = I - |w><w| where |w> is marked state
% - Single marked item among N (Grover problem)
%
% Key insight: Slow down only where gap is small
% - Gap minimum ~ 1/sqrt(N) at s* ~ 1/2
% - Adapt schedule: ds/dt ~ g(s)^2
%
% Result: T = O(sqrt(N)), matching circuit Grover
%
% Why this matters for thesis:
% - Shows AQO can match Grover
% - Chapter 5 generalizes to arbitrary diagonal H_z
% - Same idea, more complex spectrum
%
% TODO: Present Roland-Cerf as special case of AQO
% TODO: Emphasize single avoided crossing structure
%
% FORWARD: Chapter 5 generalizes this construction
%
% REVIEWER: significance - Foundation for Chapter 5

% =============================================================================
% Section: Universality
% =============================================================================
% SOURCE: Aharonov et al. (2007)
%
% AQC is computationally equivalent to circuit model:
% - Any BQP problem solvable by AQC
% - Any AQC computation simulable by circuits
% - Polynomial overhead in both directions
%
% Implications:
% - AQC is universal for quantum computation
% - No fundamental power difference from circuits
% - BUT: different resource requirements, different optimizations
%
% TODO: Cite Aharonov et al. properly
% TODO: Explain polynomial equivalence briefly
%
% OPTION: Could be brief or detailed depending on thesis scope
%
% REVIEWER: literature - Must cite universality result

% =============================================================================
% Section: Why AQO is More Restricted
% =============================================================================
% SOURCE: paper/v3-quantum.tex context
%
% AQO vs general AQC:
%
% AQO restrictions:
% - H_z is diagonal in computational basis: H_z = sum_x f(x)|x><x|
% - No entanglement in H_z (classical energies)
% - H_0 is projector: H_0 = -|psi_0><psi_0|
% - Only schedule s(t) is adjustable
%
% General AQC:
% - H_z can be any Hamiltonian
% - Can have entangled ground states
% - H_0 can be any Hamiltonian with known ground state
%
% Why study AQO?
% - Natural for combinatorial optimization (SAT, MAX-CUT, etc.)
% - Simpler spectral structure (single avoided crossing)
% - Clean analysis possible
% - Still captures essence of adiabatic speedup question
%
% TODO: Explain diagonal H_z constraint
% TODO: List problems encodable in AQO (Chapter 2/8 connection)
%
% FORWARD: Chapter 5 formalizes this setup

% =============================================================================
% Section: The No-Free-Lunch Reality
% =============================================================================
%
% Common misconception: "Just let nature compute, no algorithm needed"
%
% Reality:
% - Slowness is fundamental: gap determines runtime
% - Cannot just "run slow" uniformly - exponential overhead
% - Optimal schedule requires gap information
% - Gap information is computationally expensive (NP-hard, Chapter 8)
%
% The tension:
% - AQO can achieve Grover speedup (Chapter 7)
% - BUT: achieving it requires solving hard problem first (Chapter 8)
% - "Optimality with limitations"
%
% Preview of main results:
% 1. Optimal runtime T = O~(sqrt(N/d_0)) (Chapter 7)
% 2. Achieving it requires A_1 to exponential precision (Chapter 7)
% 3. Computing A_1 is NP-hard (Chapter 8)
%
% TODO: Frame the central tension
% TODO: Set up expectations for Chapters 5-9
%
% FORWARD: Chapters 5-9 resolve this tension
%
% REVIEWER: significance - Sets up the thesis problem
