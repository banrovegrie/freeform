% Chapter 8: Hardness of Optimality
%
% OVERVIEW: This chapter proves that computing A_1 (needed for optimal schedule)
% is NP-hard at polynomial precision and #P-hard at exponential precision.
% Central result: "Optimality with limitations" - speedup contingent on solving hard problem.
%
% NEEDS: Chapter 5 (A_1 definition, s* = A_1/(A_1+1))
% NEEDS: Chapter 7 (Why A_1 precision matters for schedule)
% NEEDS: Chapter 2/3 (NP, #P complexity classes)
%
% FORWARD: Chapter 9 asks what can be achieved with partial/no information

% =============================================================================
% Section: The Hidden Requirement Made Explicit
% =============================================================================
% SOURCE: paper/v3-quantum.tex lines 376-378, 737
%
% Recap from Chapter 7:
% - Need s* to precision O(delta_s) for optimal schedule
% - delta_s = O(2^{-n/2}) for typical cases
% - s* = A_1/(A_1 + 1) depends sensitively on A_1
%
% The question (line 737):
% "How hard is it to compute A_1 to the desired additive precision?"
%
% Formal access model (line 739):
% - Input: description of Hamiltonian H (diagonal in computational basis)
% - Output: estimate of A_1(H)
% - Precision parameter: epsilon (additive error)
%
% TODO: State the question precisely as a computational problem
% TODO: Connect to local Hamiltonian problem (line 754)

% =============================================================================
% Section: The Formal Access Model
% =============================================================================
% SOURCE: paper/v3-quantum.tex lines 743-747
%
% Notation (line 743): A_1(H) for Hamiltonian H
%
% A_1(H) = (1/2^n) * sum_{k=1}^{M-1} d_k/(E_k - E_0)
%
% Problem formulation:
% - Given: description <H> of n-qubit diagonal Hamiltonian
% - Compute: A_1(H) to additive precision epsilon
%
% Two hardness results:
% 1. NP-hard for epsilon < 1/(72(n-1)) (polynomial precision)
% 2. #P-hard for epsilon = O(2^{-poly(n)}) (exponential precision)
%
% Gap between:
% - Needed precision: O(2^{-n/2}) for optimal schedule
% - NP-hard threshold: 1/poly(n)
% - Exponential gap: NP-hard even at much coarser precision than needed
%
% TODO: Explain what "description of H" means (coefficients in some basis)
%
% REVIEWER: complexity_claims - CRITICAL: precision thresholds must be exact

% =============================================================================
% Section: NP-Hardness Overview
% =============================================================================
% SOURCE: paper/v3-quantum.tex lines 749-756, Theorem 2 lines 379-389
%
% Theorem 2 (Main Result 2, lines 379-389):
% If classical procedure C_epsilon outputs A_1 to precision epsilon, then
% 3-SAT is solvable with TWO calls to C_epsilon, provided:
%   epsilon < 1/(72) * 1/(n-1)
%
% Implication: Estimating A_1 to 1/poly(n) precision is NP-hard
%
% Strategy overview (line 756):
% 1. Distinguish whether E_0 = 0 or E_0 >= mu_1 (promise problem)
% 2. This distinguishing is NP-hard (local Hamiltonian variant, line 754)
% 3. Show two A_1 queries suffice to distinguish
% 4. Reduce 3-SAT to this distinguishing problem
%
% TODO: Introduce promise problem formulation
% TODO: Connect to local Hamiltonian problem (Kempe et al.)
%
% REVIEWER: complexity_claims - Reduction must be clearly NP-hard

% =============================================================================
% Section: The Reduction Construction
% =============================================================================
% SOURCE: paper/v3-quantum.tex lines 758-817 (Lemma, lem:distinguish-ground-energy-general)
%
% Lemma 6 (Distinguishing Lemma, lines 758-817):
% Given H with promise: either (i) E_0 = 0 or (ii) mu_1 <= E_0 <= 1-mu_2
% Can distinguish with two A_1 queries if:
%   epsilon < mu_1/(6(1-mu_1)) - d_0/(6*2^n) * 1/(mu_1*mu_2)
%
% Key construction (lines 770-773):
% H' = H tensor ((1 + sigma_z)/2)
%
% Case analysis:
%
% When E_0 = 0 (lines 774-785):
% - Ground energy of H' is 0 with degeneracy d'_0 = d_0 + 2^n
% - A_1(H) - 2*A_1(H') = 0
%
% When E_0 != 0 (lines 786-803):
% - Ground energy of H' is 0 with degeneracy 2^n
% - A_1(H) - 2*A_1(H') >= mu_1/(1-mu_1) - d_0/2^n * 1/(mu_1*mu_2)
%
% Distinguishing (lines 804-816):
% - Query C_epsilon on H and H'
% - If A_1(H) - 2*A_1(H') <= 3*epsilon: conclude E_0 = 0
% - If A_1(H) - 2*A_1(H') >= threshold - 3*epsilon: conclude E_0 != 0
% - Requires 6*epsilon < threshold
%
% TODO: Explain why tensor product shifts spectrum as claimed
% TODO: Walk through calculation in detail
%
% REVIEWER: technical_soundness - Proof verification against paper

% =============================================================================
% Section: 3-SAT Reduction
% =============================================================================
% SOURCE: paper/v3-quantum.tex lines 819-859 (Theorem, thm:reduction-sat-ground-energy)
%
% Theorem (lines 823-830):
% Computing A_1 to precision epsilon < 1/(72(n-1)) for 3-local Hamiltonians is NP-hard.
%
% Reduction (lines 831-859):
% - Based on 3-SAT to MAX-2-SAT reduction (Garey et al., line 832)
% - Variables: x_0, ..., x_{n-1} and auxiliary x_n, ..., x_{n+m-1}
% - For each clause k: H_k is 3-local Hamiltonian (lines 838-843)
%
% Hamiltonian construction (lines 838-848):
% H_k = P_{not a_k} + P_{not b_k} + P_{not c_k} + P_{not x_{n+k}}
%     + P_{a_k}P_{b_k} + P_{a_k}P_{c_k} + P_{b_k}P_{c_k}
%     + P_{not a_k}P_{x_{n+k}} + P_{not b_k}P_{x_{n+k}} + P_{not c_k}P_{x_{n+k}}
%
% where P_{x_l} = (I - sigma_z^(l))/2, P_{not x_l} = (I + sigma_z^(l))/2
%
% Properties (lines 845, 849):
% - If clause k satisfied: lowest eigenvalue of H_k is 3
% - If clause k unsatisfied: lowest eigenvalue is 4
% - Total H acts on 2m + 2n qubits
%
% Applying Lemma 6 (lines 849-858):
% - mu_1 = 1/(6m), mu_2 = 1/2
% - For n + m >= 15: epsilon < 1/(72(n-1)) suffices
%
% TODO: Explain construction intuition (penalty for unsatisfied clauses)
% TODO: Verify 3-local claim
%
% REVIEWER: complexity_claims - Must match 3-SAT NP-completeness

% =============================================================================
% Section: The Precision Gap
% =============================================================================
% SOURCE: paper/v3-quantum.tex lines 752, 859-861
%
% Key observation (lines 752, 859):
% - NP-hard: precision 1/poly(n)
% - Needed: precision O(2^{-n/2})
%
% Exponential gap:
% - Even much coarser precision than needed is already NP-hard
% - No hope of efficiently approximating A_1 to required precision
%
% Note (line 861): Reduction uses 3-local Hamiltonians
% - #P-hardness proof works for 2-local Ising (stronger)
%
% TODO: Emphasize the gap: needed precision vs tractable precision
%
% REVIEWER: significance - This gap is the crux of the limitation

% =============================================================================
% Section: #P-Hardness Overview
% =============================================================================
% SOURCE: paper/v3-quantum.tex lines 863-878, Theorem 3 lines 394-397
%
% Theorem 3 (Main Result 3, lines 394-397):
% If classical algorithm C outputs A_1 exactly or to precision O(2^{-poly(n)}),
% then all degeneracies d_k can be extracted with O(poly(n)) calls to C.
%
% Why #P-hard (lines 869-873):
% - Extracting d_0 = counting ground states
% - For 3-SAT Hamiltonian: d_0 = number of satisfying assignments
% - Counting SAT solutions (#3-SAT) is #P-complete
%
% Alternative (lines 873-876):
% - All d_k allows computing IQP circuit output probabilities
% - |<0|C_IQP|0>|^2 = |1/2^n * sum_k d_k * e^{i*Delta_k}|^2
% - IQP output probability is #P-hard
%
% TODO: Define #P class if not in Chapter 2/3
% TODO: Explain why counting is harder than decision
%
% REVIEWER: complexity_claims - #P-hardness via standard reduction

% =============================================================================
% Section: The Interpolation Attack
% =============================================================================
% SOURCE: paper/v3-quantum.tex lines 878-912 (Lemma, lem:exact-degeneracy-hard)
%
% Lemma (lines 880-912):
% With O(poly(n)) exact A_1 queries, can extract all degeneracies d_k.
%
% Modified Hamiltonian (lines 885-894):
% H'(x) = H tensor I - I tensor (x/2)*sigma_+^{(n+1)}
% where sigma_+ = (I + sigma_z)/2
%
% Key observation (lines 893-899):
% A_1(H'(x)) = 1/2^{n+1} * (sum_{k>=1} d_k/Delta_k + sum_{k>=0} d_k/(Delta_k + x/2))
%
% Define f(x) = 2*A_1(H'(x)) - A_1(H) (line 898):
% f(x) = 1/2^n * sum_{k=0}^{M-1} d_k/(Delta_k + x/2)
%
% Polynomial (lines 900-904):
% P(x) = prod_{k} (Delta_k + x/2) * f(x) = polynomial of degree M-1
%
% Lagrange interpolation (lines 905-911):
% - Evaluate f(x) at M distinct points x_1, ..., x_M
% - Reconstruct P(x) via Lagrange interpolation
% - Extract d_k = 2^n * P(-2*Delta_k) / prod_{l != k}(Delta_l - Delta_k)
%
% Total queries: 2M = O(poly(n))
%
% TODO: Walk through interpolation in detail
% TODO: Explain why Delta_k are known (from H description)
%
% REVIEWER: technical_soundness - Interpolation argument must be correct

% =============================================================================
% Section: Robustness to Small Errors
% =============================================================================
% SOURCE: paper/v3-quantum.tex lines 913-917 (Paturi's lemma)
%
% #P-hardness is robust to exponentially small errors (line 913):
% - Uses Paturi's lemma on polynomial interpolation
% - Errors in A_1 of size O(2^{-poly(n)}) still allow extraction
%
% Paturi's lemma (Corollary 1 of Paturi 1992, line 916):
% - Bounds coefficient size of polynomial given bounded evaluations
% - Allows error propagation analysis
%
% TODO: State Paturi's lemma precisely if needed
% TODO: Explain error propagation in interpolation
%
% OPTION: Could defer to appendix

% =============================================================================
% Section: The Asymmetry with Circuits
% =============================================================================
% SOURCE: paper/v3-quantum.tex line 399
%
% Grover's algorithm (circuit model):
% - Achieves same O~(sqrt(N/d_0)) runtime
% - Does NOT require pre-computing s* or A_1
% - Oracle access to f(x) suffices
% - No classical pre-computation needed
%
% AQO:
% - Achieves same runtime
% - BUT requires pre-computing s* to exponential precision
% - This requires solving NP-hard problem classically
%
% Why the asymmetry?
% - Circuit model: schedule is implicit in oracle queries
% - AQO: schedule must be explicit before execution
% - Information about problem encoded differently
%
% TODO: Make the comparison precise
% TODO: Why is this a "fundamental limitation" (line 399)?
%
% REVIEWER: significance - Key conceptual contribution of the paper

% =============================================================================
% Section: Optimality with Limitations
% =============================================================================
% SOURCE: paper/v3-quantum.tex lines 399
%
% Central message: "Optimality with limitations"
%
% What is achieved:
% - AQO matches Grover speedup: O~(sqrt(N/d_0))
% - This is optimal for unstructured search
%
% The limitation:
% - Achieving optimality requires solving NP-hard problem first
% - Specifically: need A_1 to precision O(2^{-n/2})
% - But A_1 to precision 1/poly(n) is already NP-hard
%
% Interpretation:
% - AQO is "conditionally optimal"
% - Condition: prior knowledge of spectral parameters
% - This knowledge is computationally expensive to obtain
%
% Open question (line 399):
% "Can this be circumvented without access to digital quantum computer?"
%
% TODO: Frame as information-theoretic question (Chapter 9)
% TODO: Contrast with structured problems where A_1 might be easy
%
% FORWARD: Chapter 9 explores what happens with partial information
%
% REVIEWER: significance - This is the paper's main conceptual contribution
