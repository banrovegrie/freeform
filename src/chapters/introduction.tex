% Chapter 1: Introduction
%
% OVERVIEW: Sets up the thesis problem, states main results upfront,
% and provides roadmap. Written last to reflect actual content.
%
% DEPENDS ON: Knowing content of all other chapters
%
% NOTE: Introduction should create tension, state results, provide map

% =============================================================================
% Section: The Puzzle
% =============================================================================
%
% Ground states as computational answers:
% - Physical systems naturally seek low-energy configurations
% - Encode computational problem in Hamiltonian H
% - Ground state of H encodes solution
% - "Let nature compute" - attractive idea
%
% The question:
% - Can physical systems solve hard problems efficiently?
% - Specifically: can quantum systems beat classical for optimization?
%
% TODO: Build tension before resolution
% TODO: Don't answer too quickly - let reader want the answer
%
% REVIEWER: presentation_quality - Hook the reader

% =============================================================================
% Section: Unstructured Search as Canonical Target
% =============================================================================
%
% The search problem:
% - Find marked item among N items
% - No structure to exploit (oracle model)
%
% Classical: O(N) queries required
% - Must check every item in worst case
%
% Quantum: O(sqrt(N)) queries via Grover
% - Quadratic speedup is provably optimal (BBBV)
% - Cannot do better for unstructured search
%
% Why this is the right benchmark:
% - Clean problem with clear lower bound
% - Speedup is quadratic, not exponential
% - Achieved by circuit model (Grover)
% - Question: can AQO match this?
%
% TODO: Frame Grover as the benchmark to beat/match
%
% FORWARD: Chapters 3 (Grover), 7 (AQO matches it)

% =============================================================================
% Section: Adiabatic Quantum Optimization
% =============================================================================
%
% AQC idea:
% - Encode problem in Hamiltonian H_z
% - Start with easy Hamiltonian H_0 (known ground state)
% - Interpolate: H(s) = (1-s)H_0 + sH_z
% - Evolve slowly: system stays in ground state
%
% AQO = AQC with restrictions:
% - H_z is diagonal (classical energies)
% - H_0 is projector onto uniform superposition
% - Only schedule s(t) is adjustable
% - More constrained than general AQC
%
% The appeal:
% - "Nature computes for you"
% - No explicit gate sequences
% - Potentially robust to some noise
%
% The catch (preview):
% - Schedule requires knowledge of spectral gap
% - Gap information is computationally expensive
%
% TODO: Make AQO appealing before showing limitations
%
% FORWARD: Chapter 4 (AQC), Chapter 5 (AQO setup)

% =============================================================================
% Section: The Trilogy - Main Results Frontloaded
% =============================================================================
% SOURCE: paper/v3-quantum.tex Theorems 1-3
%
% The three main results (one paragraph summary):
%
% (i) Optimality exists:
% Runtime T = O~(sqrt(N/d_0)) for AQO
% Matches Grover lower bound (up to polylog factors)
% SOURCE: Theorem 1, Chapter 7
%
% (ii) Computing schedule parameters is NP-hard:
% Estimating A_1 to 1/poly(n) precision reduces 3-SAT
% Only two oracle calls needed
% SOURCE: Theorem 2, Chapter 8
%
% (iii) Exact parameter values are #P-hard:
% Computing A_1 exactly enables extracting all degeneracies
% Polynomial interpolation attack
% SOURCE: Theorem 3, Chapter 8
%
% The tension:
% - Optimal speedup exists (theoretically)
% - But achieving it requires solving NP-hard problem first
% - "Optimality with limitations"
%
% TODO: State results crisply without full detail
% TODO: Create tension: optimality exists but is contingent
%
% REVIEWER: significance - This is the paper's contribution

% =============================================================================
% Section: The Central Question
% =============================================================================
%
% When does "nature computes for you" actually work?
%
% The naive view:
% - Set up Hamiltonian, wait, measure
% - Nature does the optimization
% - No algorithm design needed
%
% The reality:
% - Must choose schedule s(t)
% - Optimal schedule requires gap information
% - Gap information is NP-hard to compute
%
% Reframing the question:
% - Not "can AQO achieve speedup?" (yes, in principle)
% - But "when is the gap information tractable?"
% - Information as the bottleneck, not computation
%
% This thesis:
% - Proves the optimality/hardness results (Chapters 5-8)
% - Characterizes the information-runtime tradeoff (Chapter 9)
% - Provides machine-checked verification (Chapter 10)
%
% TODO: Frame as information question
% TODO: Set up Chapter 9 extensions
%
% FORWARD: Chapter 9 (information gap)

% =============================================================================
% Section: Thesis Contributions
% =============================================================================
%
% What this thesis contributes:
%
% 1. Exposition of the published work:
%    - Deep explanation of paper results (Chapters 5-8)
%    - More pedagogical than original paper
%    - Connects to broader context
%
% 2. Original extensions (Chapter 9):
%    - Separation theorem: informed vs uninformed
%    - Partial information tradeoff
%    - Measure condition and scaling spectrum
%    - Robust and adaptive schedules
%
% 3. Formalization (Chapter 10):
%    - Lean 4 verification of results
%    - Machine-checked proofs
%    - Errors discovered and fixed
%
% The goal:
% - Best single source for understanding this topic
% - Clear enough to teach
% - Rigorous enough to build on
%
% TODO: Be honest about what is exposition vs original
%
% REVIEWER: originality - Clear about contributions

% =============================================================================
% Section: Chapter Overview and Reading Paths
% =============================================================================
%
% Chapter structure:
%
% Background (can skim if familiar):
% - Chapter 2: Physics and Computation - first principles
% - Chapter 3: Quantum Computation - QM, Hilbert spaces, Grover
% - Chapter 4: Adiabatic Quantum Computation - AQC framework
%
% Core paper results (essential):
% - Chapter 5: Adiabatic Quantum Optimization - AQO setup, spectral parameters
% - Chapter 6: Spectral Analysis - gap bounds in three regions
% - Chapter 7: Optimal Schedule - runtime derivation, Grover matching
% - Chapter 8: Hardness of Optimality - NP-hard, #P-hard results
%
% Extensions (original contributions):
% - Chapter 9: Information Gap - information-runtime tradeoffs
%
% Verification:
% - Chapter 10: Formalization - Lean proofs, errors caught
%
% Conclusion:
% - Chapter 11: Summary, open problems, future directions
%
% Reading paths:
% - QC expert: skim 2-3, read 4-11
% - Complexity theorist: read 2-8, skim 9-10
% - Full reading: 1-11 in order
%
% TODO: Match paths to reader backgrounds
%
% REVIEWER: presentation_quality - Roadmap helps navigation
