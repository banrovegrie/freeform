% Chapter 11: Conclusion
%
% OVERVIEW: Synthesizes thesis results, discusses implications,
% lists open problems. Written last to reflect actual content.
%
% DEPENDS ON: Knowing content of all other chapters

% =============================================================================
% Section: The Trilogy Revisited
% =============================================================================
% SOURCE: Chapters 5-8
%
% One-page summary of main results:
%
% Main Result 1 (Chapter 7):
% AQO achieves runtime T = O~(sqrt(N/d_0)) for diagonal Hamiltonians
% satisfying the spectral condition.
% - Matches Grover lower bound up to polylog factors
% - Optimal for unstructured search
% - Requires local adaptive schedule
%
% Main Result 2 (Chapter 8):
% Computing A_1 to precision 1/poly(n) is NP-hard.
% - Reduces 3-SAT with two oracle calls
% - Even coarse approximation is intractable
%
% Main Result 3 (Chapter 8):
% Computing A_1 exactly is #P-hard.
% - Polynomial interpolation extracts all degeneracies
% - d_0 extraction solves #SAT
%
% The synthesis:
% - Optimality exists but requires solving NP-hard problem
% - "Optimality with limitations" captures this tension
%
% TODO: Crisp one-paragraph per result
%
% REVIEWER: presentation_quality - Clear summary

% =============================================================================
% Section: The Conceptual Takeaway
% =============================================================================
%
% "Optimality with limitations" as a paradigm:
%
% What we learned:
% - Information determines speedup, not just complexity class
% - The gap is both physical bottleneck and information barrier
% - Same speedup (Grover) has different costs in different models
%
% The gap as bridge:
% - Physics: gap is energy difference, determines adiabatic condition
% - Computation: gap determines runtime, encodes hardness
% - Information: gap parameters are NP-hard to compute
%
% New perspective on AQO:
% - Not "does AQO work?" but "when is gap information tractable?"
% - Shifts focus from speedup existence to information requirements
%
% TODO: Frame as paradigm shift
%
% REVIEWER: significance - Conceptual contribution matters

% =============================================================================
% Section: Comparison with Circuit Model
% =============================================================================
%
% AQO vs Grover circuit:
%
% Same speedup: O(sqrt(N/d_0))
%
% Different information requirements:
% - Grover: oracle access to f(x), no pre-computation needed
% - AQO: requires schedule parameters (A_1) to exponential precision
%
% The asymmetry:
% - Grover queries oracle during execution
% - AQO needs gap information before execution
% - Why does same speedup have different costs?
%
% Interpretation:
% - Circuit model: information gathered during computation
% - Adiabatic model: information needed upfront for schedule
% - Different models, different information economics
%
% TODO: Make asymmetry vivid
%
% REVIEWER: significance - Important conceptual point

% =============================================================================
% Section: What This Changes
% =============================================================================
%
% Implications for AQO research:
%
% Evaluating proposals:
% - Don't just ask "what is the gap?"
% - Also ask "how hard is computing the gap?"
% - Precision as a resource
%
% The new question:
% - From "can AQO be fast?" to "when is A_1 tractable?"
% - Structured instances may have efficient parameter computation
% - Problem structure determines information tractability
%
% Experimental implications:
% - Scheduling requires either knowing A_1 or adaptive measurement
% - Quantum annealing in practice: what schedule is actually used?
% - Deviations from optimal may be fundamental, not engineering
%
% TODO: Connect to practical AQO/annealing
%
% REVIEWER: significance - Practical relevance

% =============================================================================
% Section: Original Contributions (Chapter 9)
% =============================================================================
%
% Summary of extensions:
%
% 1. Separation Theorem:
%    - T_uninformed / T_informed >= (s_R - s_L) / Delta_*
%    - Omega(2^{n/2}) separation for unstructured search
%    - First minimax lower bound for fixed uninformed schedules
%    - Lean formalization: COMPLETE
%
% 2. Partial Information Tradeoff:
%    - T(epsilon) = T_inf * Theta(max(1, epsilon/delta_A1))
%    - Linear interpolation between informed and uninformed
%    - No threshold effects, smooth degradation
%    - Lean formalization: COMPLETE
%
% 3. Measure Condition and Scaling:
%    - T = Theta(1/Delta_*^{3-2/alpha}) for gap approach exponent alpha
%    - Dichotomy conjecture FALSE: continuous spectrum of scalings
%    - Lean formalization: COMPLETE
%
% 4. Robust Schedules:
%    - Hedging with bounded uncertainty achieves constant-factor
%    - Softens NP-hardness barrier
%    - Lean formalization: PARTIAL
%
% 5. Adaptive Schedules:
%    - O(n) measurements suffice for optimal runtime
%    - Circumvents classical hardness
%    - Lean formalization: PARTIAL
%
% The ignorance taxonomy:
% - No information: exponential overhead
% - Intermediate precision: polynomial overhead (linear in epsilon)
% - Bounded interval: constant factor
% - Quantum measurement: logarithmic (O(n) measurements)
%
% TODO: Connect back to central thesis
%
% REVIEWER: originality - These are novel contributions

% =============================================================================
% Section: The Formalization Standard (Chapter 10)
% =============================================================================
%
% What formalization achieved:
%
% 1. Error detection:
%    - 5+ formulation issues discovered
%    - Sign errors, missing hypotheses, direction reversals
%    - Formalization as debugging tool
%
% 2. Verification:
%    - 27 axioms, 76+ theorems, 0 sorries
%    - Machine-checked proofs
%    - Explicit trust boundary (axioms)
%
% 3. Reproducibility:
%    - Lean code compiles = proofs valid
%    - Independent verification possible
%    - Future-proof documentation
%
% The axiom boundary:
% - External foundations (Cook-Levin, Valiant, adiabatic theorem)
% - Gap bounds (require parametric spectral theory)
% - Novel mathematics: fully proved
%
% TODO: Advocate for formalization in QC
%
% REVIEWER: significance - Formalization contribution

% =============================================================================
% Section: Open Problems
% =============================================================================
%
% Research directions:
%
% 1. Structured instances:
%    - When is A_1 efficiently computable?
%    - Problem structure -> information tractability
%    - Examples: planted SAT, MAX-CUT on specific graphs
%
% 2. Noise models:
%    - How does decoherence affect information-runtime tradeoffs?
%    - Does noise help or hurt AQO relative to circuits?
%    - Open-system adiabatic evolution
%
% 3. Intermediate precision:
%    - What can polynomial precision achieve?
%    - Are there useful regimes between 1/poly(n) and 2^{-n/2}?
%    - Partial structure exploitation
%
% 4. Alternative paths:
%    - Non-linear interpolations: H(s) with different s-dependence
%    - Catalyst Hamiltonians: adding terms to improve gap
%    - Counterdiabatic driving
%
% 5. Quantum annealing connection:
%    - Real devices use thermal fluctuations
%    - What does our analysis say about practical annealing?
%    - Theory-experiment gap
%
% 6. The ultimate question:
%    - Is there a family of problems where AQO beats circuits?
%    - Not just matches Grover, but exceeds circuit lower bounds
%    - Would require structured instances with tractable A_1
%
% TODO: Be specific about what is open
%
% REVIEWER: significance - Guide future work

% =============================================================================
% Section: Final Remarks
% =============================================================================
%
% What we set out to do:
% - Explain the UAQO paper deeply
% - Place it in broader context
% - Extend with original contributions
% - Verify with machine-checked proofs
%
% What we achieved:
% - Complete exposition of optimality/hardness results
% - Information-theoretic characterization of the tradeoffs
% - Lean formalization with errors caught and fixed
% - A unified perspective: information determines speedup
%
% The central message:
% - AQO can match Grover speedup
% - But optimality requires information that is NP-hard to obtain
% - This is "optimality with limitations"
% - The gap bridges physics, computation, and information
%
% For the reader:
% - You now understand what AQO can and cannot do
% - You have tools to analyze specific instances
% - You know the open questions
%
% TODO: End with the central insight
%
% REVIEWER: presentation_quality - Strong closing
